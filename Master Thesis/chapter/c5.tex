\documentclass{standalone}
% preamble: usepackage, etc.
\begin{document}
\chapter{Conclusion}
\label{Chapter5}
Due to the obvious increased output of large amounts of information in recent times, particularly images, it is becoming necessary to build techniques and software for the automatic collection, processing, and analysis of such information in order to extract trends and patterns. This guarantees that images and its characteristics are converted by algorithm into representations that reflect their context and interpretation. Currently, innovations in artificial intelligence are being applied to system algorithms for image analysis and complicated analysis in the field of machine vision. Image classification and recognition are essential components of machine vision. These activities are simple for humans, but they require complicated numerical concepts and advanced computational procedures for machines to execute. This dissertation presents wavelet multi-resolution analysis and deep learning model combined into a single framework for image recognition and classification, with up-to-date results on the MIT-BIH normal sinus rhythm, case western reserve university, COVID-19 radiograph database, CICDDos2019, NIH pneumonia, RSNA pneumonia and CT-datasets, respectively.


\textbf{\section{Research Findings}}
In this dissertation, we examined the application of wavelet multi-resolution analysis and deep learning models as a single framework for the task of machine fault diagnosis, biometric identification and disease classification. We recognize the importance of decreasing the discrepancy between computer and human judgment, therefore we have designed models that contributes to the advancement of artificial intelligence, particularly in machine vision.

\textbf{\section{Contributions}}
In this dissertation, we presented wavelet multi-resolution analysis algorithm and deep learning model as a single framework capable of attaining and advancing the horizon of research in medical imaging, machine fault diagnosis and biometric identification. We believe that this advances the requirement for intelligent data analysis, interpretation, and augmenting a limited human reliance on image interaction and understanding.


\textbf{\section{Future Works}}
Given the vast amount of research in vision recognition and image classification, there are still a number of obstacles to overcome before reaching same accuracy of that of human. To begin, our models are supervised framework, which necessitates a large quantity of data to learn the mapping from an input $X$ to a label $Y$. This type of learning domain necessitates a lot of human tagging and labeling, which makes the process difficult and resource-intensive from a computational and hardware standpoint. Designing models that acquire intuitive representation without the need for limitless training sets is a viable solution.


\end{document}